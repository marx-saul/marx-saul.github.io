\documentclass{jsarticle}

\usepackage{amsmath, amssymb, amsthm}
\usepackage{enumerate}

\theoremstyle{definition}
\newtheorem*{axiom*}{公理}
\newtheorem*{definition*}{定義}
\newtheorem{theorem}{定理}[section]
\newtheorem{proposition}[theorem]{命題}
\newtheorem{lemma}[theorem]{補題}
\newtheorem{corollary}[theorem]{系}
\newtheorem{remark}{注}[section]
\newtheorem*{remark*}{注}
\newtheorem{example}{例}[section]
\renewcommand\proofname{\rm [証明]}

\parindent = 0pt

\begin{document}
    \section{整列順序}
    \begin{definition*} \label{well_founded_order}
        $\langle X, < \rangle$ が整列順序(well-founded order)であるとは、
        \begin{enumerate}[(i)]
            \item $\lnot x < x$
            \item $x < y \land y < z \rightarrow x < z$
            \item $x < y \lor x = y \lor y < x$
            \item $A \subseteq X$ が空でないなら $A$ の最小元が存在する
        \end{enumerate}
        を満たすことをいう。
    \end{definition*}
    
    \begin{definition*}
        $\langle X, < \rangle$ が整列順序であるとき、$x \in X$ に対して $\operatorname{pred}(X, x) = \{a \in X : a < x\}$ とする。$<$ をこの上に制限した整列順序で整列づけされるものとする。
    \end{definition*}

    \begin{definition*}
        $\langle X, < \rangle, \, \langle Y, <' \rangle$ が整列順序であるとき、$f : X \rightarrow Y$ が順序を保つ写像であるとは、$x < x' \rightarrow f(x) < f(x')$ が成り立つことをいう。順序同型写像とは、順序を保つ全単射のことをいう。$\langle X, < \rangle, \, \langle Y, <' \rangle$ の間に順序同型写像が存在するとき、順序同型といい、$X \simeq Y$ と表す。
    \end{definition*}

    \begin{theorem} \label{pred_not_isomorphic}
        $\langle X, < \rangle$ を整列順序として、$x \in X$ とする。$\operatorname{pred}(X, x) \not\simeq X.$
    \end{theorem}
    \begin{proof}
        $f : X \rightarrow \operatorname{pred}(X, x)$ が順序同型であるとする。帰納法で $a \in X \rightarrow a \leq f(a)$ が示せる。実際、$b < a \rightarrow b \leq f(b)$ であるとする。$f(a) < a$ なら、仮定より $f(f(a)) \leq f(a).$ 一方 $f$ は順序を保つから $f(f(a)) < f(a)$ で矛盾。よって、$a \leq f(a)$ で、ok.\\
        $x \leq f(x)$ で、矛盾。
    \end{proof}

    \begin{theorem} \label{well_found_theorem}
        $\langle X, < \rangle, \, \langle Y, <' \rangle$ を整列順序とすると、以下のただひとつだけが成立する :
        \begin{enumerate}[(i)]
            \item $X \simeq Y$
            \item $\operatorname{pred}(X, x) \simeq Y$ となる $x \in X$ がある
            \item $X \simeq \operatorname{pred}(Y, y)$ となる $y \in Y$ がある
        \end{enumerate}   
    \end{theorem}
    \begin{proof}
        二つ以上が同時に成立することはないのは前定理を使えばすぐに示せる。
        \begin{alignat*}{5}
           X' &=& \{x &\in& X : \exists y &\in& Y (\operatorname{pred}(X, x) &\simeq& \operatorname{pred}(Y, y))\}&\\
           Y' &=& \{y &\in& Y : \exists x &\in& X (\operatorname{pred}(Y, y) &\simeq& \operatorname{pred}(X, x))\}&\\
        \end{alignat*}
        とおくと、$X'$ は $X$ それ自身か、$X$ の切片であることが示せて、前定理より $X' \simeq Y'$ であることが示せる。$X' = \operatorname{pred}(X, x), Y' = \operatorname{pred}(Y, y)$ がどちらも切片なら $x \in X'$ となるはずで、矛盾。よってどれか一つは成立する。
    \end{proof}
    \newpage
    
    \section{順序数}
    \begin{definition*} \label{transitive}
        集合 $x$ が推移的(transitive)であるとは、$\forall y \in x \, (y \subseteq x)$ となることをいう。
    \end{definition*}
    \begin{definition*} \label{ordinal_number}
        集合 $x$ が順序数(ordinal number)であるとは、推移的かつ、$R = \{ \langle y, z \rangle \in x \times x : y \in z\}$ が $x$ を整列付けすることをいう。
    \end{definition*}
    \begin{example}
        $0 = \{\}, \ 1 = \{0\}, \ 2 = \{0, 1\}, \cdots$
    \end{example}
    \begin{remark}
        整列順序の同型定理(\ref{well_found_theorem}) によると、整列順序は、「整列集合の大きさ」以外に違いがないことがわかる。ところで、$\in$ は二項関係に見えるから、これを整列順序に持つような整列集合を考えるとなにか面白い事が言えそう。それだけだとうまく行かないから、「推移的」という仮定を加えた。
    \end{remark}
    \vspace{1ex}
    
    \begin{proposition} \label{ordinal_basic_1} \
        \begin{enumerate}[(i)]
            \item $\beta$ が順序数であり、$\alpha \in \beta$ なら $\alpha$ も順序数である。
            \item $\alpha, \beta$ が順序数で、$\alpha \simeq \beta$ なら $\alpha = \beta.$
            \item $\alpha, \beta$ が順序数なら $\alpha \in \beta, \ \alpha = \beta, \ \beta \in \alpha$ のどれかひとつだけが成立する。
        \end{enumerate}   
    \end{proposition}
    \begin{proof} \ \\
        (i): \\
        $\alpha \subseteq \beta$ だから、$\in$ によって整列付けされることは明らか。推移的であることを示せば良い。$\gamma \in \alpha, \ \delta \in \gamma$ とする。\\
        $\gamma \in \alpha \subseteq \beta$ だから $\gamma \in \beta.$ よって $\gamma \subseteq \beta.$ したがって、$\delta \in \gamma \subseteq \beta$ なので $\delta \in \beta.$ \\
        $\alpha, \gamma, \delta \in \beta$ であり、$\delta \in \gamma \in \alpha$ だから $\delta \in \alpha.$ よって $\gamma \subseteq \alpha$ であり、$\alpha$ は推移的である。\\
        (ii): \\
        $f: \alpha \rightarrow \beta$ が順序同型であるとする。帰納法で、$\xi \in \alpha$ に対して $f(\xi) = \xi$ であることを示す。$\eta \in \xi \rightarrow \eta = f(\eta)$ を仮定する。順序を保つから $\eta \in \xi \rightarrow \eta = f(\eta) \in \xi.$ よって $\xi \subseteq f(\xi).$ 逆に、$\eta \in f(\xi)$ なら、$f(\eta') = \eta$ となる $\eta'$ は、$\eta' \in \xi$ であるので、$\eta = f(\eta') = \eta'$ であり、$\eta \in \xi.$ つまり $f(\eta) \subseteq \eta.$\\
        (iii): \\
        $\xi$ の $\eta$ による切片は $\eta$ 自身になることと、(ii) と (\ref{well_found_theorem}) を用いれば良い。\\
    \end{proof}
    \vspace{1ex}
    
    \begin{remark*}
        ギリシャ文字 $\alpha, \beta, \cdots$ は特に断らなくても順序数のこととする。$\forall \alpha (\cdots)$ などは $\forall x (x$ は順序数 $\rightarrow \cdots)$ の略記とする。\\
        また、$\alpha < \beta$ は $\alpha \in \beta$ のこととする。
    \end{remark*}
    
    \begin{definition*} \
        $\bf{ON} = \{\ \alpha : \alpha$ は順序数 $\}.$
    \end{definition*}
    \begin{proposition} $\bf{C} \subseteq \bf{ON}$ が空でないクラスなら、最小元を持つ。
    \end{proposition}
    \begin{proof}
        $\alpha \in \bf{C}$ が最小元でないとき、$\alpha$ の空でない部分集合 $\alpha \cap \bf{C}$ の最小元は $\bf{C}$ の最小元である。
    \end{proof}
    \begin{lemma} \label{ordinal_basic_2} $C$ が順序数からなる推移的な集合ならば、順序数である。
    \end{lemma}
    \begin{proof}
        $\in$ は $C$ を整列順序付けする。
    \end{proof}
    \begin{corollary} \label{Burali_Forti_Paradox} (ブラリ=フォルティのパラドックス) \\
    $\bf{ON}$ は真のクラスである。
    \end{corollary}
    \begin{proof}
        集合 $ON = \{\ \alpha : \alpha$ は順序数 $\}$ が存在したとする。$ON$ は順序数からなる推移的な集合だから、前定理より $ON$ は順序数。よって $ON \in ON$ だが、(\ref{ordinal_basic_1}-iii) より、自身を含む順序数は存在しない。
    \end{proof}
    \vspace{1ex}
    
    \begin{theorem}
        $\langle X, R \rangle$ を整列順序とすると、それと順序同型な順序数がただ一つ存在する。
    \end{theorem}
    \begin{proof}
        $p \not\in X$ を適当にとって $X' = X \cup \{p\}$ に整列順序 $<'$ を\\
        $x <' y \leftrightarrow (x, y \in X \land x < y) \lor (x \in X \land y = p)$ と定義すれば、$X = \operatorname{pred}(X', p)$ である。\\
        そこで、$x \in X$ の帰納法で、$\operatorname{pred}(X, x)$ と順序同型になる順序数の存在が言えれば良い(唯一性は、(\ref{ordinal_basic_1}-ii)より自動的にしたがう)。\\
        $x' < x$ のとき $\operatorname{pred}(X, x')$ と順序同型になる(唯一の)順序数を $\alpha_{x'}$ として、置換公理によって $C = \{\alpha_{x'} : x' < x\}$ が存在する。\\
        このとき、$C$ は順序数であり、$\operatorname{pred}(X, x) \simeq C$ となることを示そう。$\beta \in \alpha_{x'}$ のとき、$\beta$ は $\alpha_{x'}$ の切片であり、順序同型写像 $f_{x'} : \operatorname{pred}(X, x') \rightarrow \alpha_{x'}$ による $\beta$ の逆像は再び切片であり、よって $\beta \in C.$ よって $C$ は順序数からなる推移的な集合だから、順序数。$f : \operatorname{pred}(X, x) \rightarrow C$ を $x' \mapsto \alpha_{x'}$ で定めればこれは順序同型である。
    \end{proof}
    \begin{definition*}
        $\langle X, R \rangle$ が整列順序のとき、$\operatorname{type}(X, R)$ を $\langle X, R \rangle$ と順序同型な(ただ一つの)順序数とする。
    \end{definition*}
    \vspace{1ex}
    
    \begin{lemma} \label{ordinal_basic_3} $\alpha \leq \beta \rightarrow \alpha \subseteq \beta$ である。
    \end{lemma}
    \begin{proof}
        $\alpha \in \beta$ なら $\alpha \subseteq \beta.$ $\alpha = \beta$ なら $\alpha \subseteq \beta.$ よって $\alpha \leq \beta \rightarrow \alpha \subseteq \beta.$ \\
        $\alpha \subseteq \beta \land \beta < \alpha$ なら $\beta \in \beta$ となって矛盾するから、$\alpha \subseteq \beta \rightarrow \alpha \leq \beta$
    \end{proof}
        
    \begin{lemma} \label{ordinal_basic_4} $C$ が順序数からなる集合のとき、$\sup C := \bigcup C$ は順序数であり、$C$ の上限である。
    \end{lemma}
    \begin{proof}
        $\alpha \in \bigcup C$ のとき、$\alpha \in \beta$ となるような $\beta \in C$ がある。(\ref{ordinal_basic_1}-i) より $\alpha$ は順序数。\\
        $\gamma \in \alpha$ のとき、$\alpha \subseteq \beta$ より $\gamma \in \beta \subseteq \bigcup C$ だから、$\alpha \subseteq \bigcup C.$\\
        よって $\bigcup C$ は順序数からなる推移的な集合だから、順序数である。\\
        $\xi$ が $C$ の上界なら、$\alpha \in C \rightarrow \alpha \subseteq \xi$ ゆえ $\bigcup C \subseteq \xi.$ \\
        $\alpha \in C \rightarrow \alpha \subseteq \bigcup C$ は明らか。よって上限であることが示せた。 
    \end{proof}
    \vspace{1ex}
    
    \begin{definition*} \label{successor} $S(x) := x \cup \{x\}.$
    \end{definition*}
    \begin{remark}
        以下に証明するように、$S(\alpha)$ は $\alpha$ よりも大きい順序数のうち最小の順序数である。
    \end{remark}
    
    \begin{lemma} \label{ordinal_basic_5} \
        \begin{enumerate}[(i)]
            \item $S(\alpha)$ は順序数である。
            \item $\alpha < \beta \leftrightarrow \alpha \leq \beta.$
            \item $\alpha < S(\beta) \leftrightarrow \alpha \leq \beta.$
        \end{enumerate} 
    \end{lemma}
    \begin{proof} \ \\
        (i): (\ref{ordinal_basic_4}) を使う。\\
        (ii): $\alpha \in \beta \leftrightarrow S(\alpha) \subseteq \beta.$\\
        (iii): $\alpha \in S(\beta) \leftrightarrow \alpha \in \beta \lor \alpha \in \{\beta\}.$
    \end{proof}
    \vspace{0.5ex}
    
    \begin{definition*} \label{zero_one_} $0 = \{\}, \ 1 = S(0), \ 2 = S(1), \ 3 = S(2),\  \cdots$
    \end{definition*}
    \begin{definition*} \label{limit_ordinal} $S(\alpha') = \alpha$ となるような $\alpha'$ が存在するとき、$\alpha$ を後続順序数という。$0$ でも後続順序数でもない順序数を 極限順序数という。
    \end{definition*}
    \begin{definition*} \label{natural_number} $\alpha$ が自然数であるとは、$\forall \beta \leq \alpha (\,\beta = 0 \lor \beta$ は後続順序数 $)$ となることをいう。
    \end{definition*}
    
    ここで、無限公理を思い出す:
    \begin{axiom*} (\bf Inf)
        \begin{center}
            $\exists X (0 \in X \land \forall x \in X(S(x) \in X))$
        \end{center}
    \end{axiom*}
    
    このような $X$ は任意の自然数を含む。実際、$n$ が $X$ に含まれない最小の自然数なら、$n > 0$ ゆえ、$n = S(m)$ となる $m < n$ がある。$m$ は自然数であるから、最小性より $m \in X$ ゆえ $n = S(m) \in X$ となって矛盾する。そこで、以下の定義ができる:
    
    \begin{definition*} \label{omega} $\omega = \{\ n : n$ は自然数 $\}$
    \end{definition*}
    
    \begin{theorem} \label{Peano_axioms} (ペアノの公理) \\
        $\omega$ は最小の極限順序数であり、以下を満たす:
        \begin{enumerate}[(i)]
            \item $0 \in \omega$
            \item $\forall n \in \omega (S(n) \in \omega)$
            \item $\lnot\exists n \in \omega (S(n) = 0)$
            \item $\forall n \in \omega \forall m \in \omega (S(n) = S(m) \rightarrow n = m)$
            \item $\forall X \subseteq \omega (0 \in X \land \forall n \in X (S(n) \in X) \rightarrow X = \omega)$
        \end{enumerate}
    \end{theorem}
    \begin{proof} \ \\
    (v): $\omega$ の定義の前にある説明と同じ方法で示される。
    \end{proof}
    \newpage
    
    \section{超限再帰}
    \begin{definition*} \label{transfinite_recursion}
        ${\bf V} = \{ x : x = x \}.$
    \end{definition*}
    \begin{theorem}
        ${\bf F} : {\bf V} \rightarrow {\bf V}$ を関数クラスとする。このとき、唯一の関数クラス ${\bf G} : \bf{ON} \rightarrow {\bf V}$ で \\
        $$\forall \alpha(\,{\bf G}(\alpha) ={\bf F}({\bf G}\restriction\alpha)\,)$$
        を満たすものが存在する。
    \end{theorem}
    \begin{proof} \ \\
        唯一性はほとんど自明で、${\bf G}_1, {\bf G}_2$ がそれなら、$\alpha < \delta \rightarrow {\bf G}_1(\alpha) = {\bf G}_2(\alpha)$ が成立するとき ${\bf G}_1\restriction\delta = {\bf G}_2\restriction\delta$ ゆえ ${\bf G}_1(\alpha) = {\bf F}({\bf G}_1\restriction\delta) = {\bf F}({\bf G}_2\restriction\delta) = {\bf G}_2(\alpha).$ よって帰納法によって一意性は示されている。\\
        存在性について、$\delta$ 上の関数 $g$ が $\forall \alpha < \delta(\,g(\alpha) ={\bf F}(g\restriction\alpha)\,)$ を満たしているとき、$g$ を $\delta$-近似と呼ぶことにすれば、$\delta$ 近似は一意的に存在し、$\delta' < \delta$ のとき $g\restriction\delta'$ は $\delta'$-近似になる。\\
        $0$-近似が存在するのは自明。\\
        $\alpha < \delta$ のとき $\alpha$-近似 $g_\alpha$ が存在するとしよう。$\delta = S(\delta')$ となるような $\delta'$ があるとき、$g_\delta$ を、$\alpha < \delta'$ のとき $g_\delta(\alpha) = g_{\delta'}(\alpha), \ g_\delta(\delta') = {\bf F}(g_{\delta'})$ とすれば、$g_\delta$ は $\delta$-近似である。\\
        $\delta$ が極限順序数であるなら、$\alpha < \delta$ に対して、適当な $\beta \ (\alpha < \beta < \delta)$ で $g_\delta(\alpha) = g_\beta(\alpha)$ と定義すれば、$g_\delta$ は $\delta$-近似である。\\
        最後に、${\bf G}(\alpha)$ を、適当な $\delta$-近似 $(\delta > \alpha)$ $g_\delta$ による $g_\delta(\alpha)$ と定義すれば、これは定理の主張を満たす。
    \end{proof}
    
    \begin{example}
        ${\bf F} : {\bf V} \rightarrow {\bf V}$ を、$f$ がある $n \in \omega$ 上の関数であるとき、$n > 0 \rightarrow {\bf F}(f) = n \times f(n-1), \ n = 0 \rightarrow {\bf F}(f) = 1$ と定義して、それ以外の場合 ${\bf F}(f) = 0$ としておく。このとき、上の定理が主張する ${\bf G}$ は $\omega$ に制限したとき、階乗関数そのものである。
    \end{example}
    
    \begin{example} \ \\
        $\gamma$ を固定して
        \begin{itemize}
        \item $\gamma + 0 = \gamma,$
        \item $\gamma + S(\alpha) = S(\gamma + \alpha),$
        \item $\alpha$ が極限順序数なら $\gamma + \alpha = \sup \{\gamma + \xi : \xi < \alpha\},$
        \end{itemize}
        と帰納的に定める。また、
        \begin{itemize}
        \item $\gamma \cdot 0 = 0,$
        \item $\gamma \cdot S(\alpha) = \gamma \cdot \alpha + \gamma,$
        \item $\alpha$ が極限順序数なら $\gamma \cdot \alpha = \sup \{\gamma \cdot \xi : \xi < \alpha\},$
        \end{itemize}
        \begin{itemize}
        \item $\gamma^0 = 0,$
        \item $\gamma^S(\alpha) = \gamma^\alpha \cdot \gamma,$
        \item $\alpha$ が極限順序数なら $\gamma^\alpha = \sup \{\gamma^\xi : \xi < \alpha\},$
        \end{itemize}
        と帰納的定義する。\\
        この定義は、以下の組み合わせ的定義と一致する : \\
        $\alpha + \beta = \operatorname{type}(\alpha \times \{0\} \cup \beta \times \{1\}, \ R).$ ただし、
        \begin{alignat*}{1}
            R = \, & \{\langle \langle \xi, 0 \rangle, \langle \eta, 0 \rangle \rangle : \xi, \eta \in \alpha, \ \xi < \eta\}\\
            \cup \,& \{\langle \langle \xi, 1 \rangle, \langle \eta, 1 \rangle \rangle : \xi, \eta \in \beta, \ \xi < \eta\}\\
            \cup \, & \{\langle \langle \xi, 0 \rangle, \langle \eta, 1 \rangle \rangle : \xi \in \alpha, \ \eta \in \beta\}.
        \end{alignat*}
        $\alpha \cdot \beta = \operatorname{type}(\beta \times \alpha, \lhd).$ ただし、
        \begin{alignat*}{1}
            &\langle \xi, \eta \rangle \lhd \langle \xi', \eta' \rangle \leftrightarrow (\xi < \xi') \lor (\xi = \xi' \land \eta < \eta')
        \end{alignat*}
    \end{example}
    
    \section{基数}
    \begin{definition*}
        $X \preceq Y$ とは $X$ から $Y$ への単射が存在すること、$X \approx Y$ とは $X$ から $Y$ への単射が存在すること、そして $X \prec Y$ とは $X \preceq Y \land X \not\approx Y$ のことと定義する。
    \end{definition*}
    \begin{theorem} \label{Bernstein} (ベルンシュタインの定理) \\
        $X \preceq Y \land Y \preceq X \rightarrow X \approx Y.$ 
    \end{theorem}
    \begin{theorem} \label{Cantor} (カントールの定理) \\
        $X \prec \mathcal{P}(X).$ 
    \end{theorem}
    \vspace{1ex}
    
    \begin{definition*}
        $X \approx \alpha$ となるような $\alpha$ のうち最小のものを $|X|$ と書く。
    \end{definition*}
    \begin{remark}
        $|X|$ が定義できるためには、$X \approx \alpha$ となるような順序数が少なくとも一つは存在しなければならない。これは、$X$ が整列順序付け可能であることと同値である。$\rm AC$ のもとでは全ての集合が整列順序付け可能であるから、$|X|$ が常に定義できる。しかし、$\rm AC$ を仮定せずに話を進めるときは、{\bf $|X|$ と書いたら $X$ は整列順序付けできるという仮定を暗に含んでいる}ことにする。
    \end{remark}
    \begin{definition*}
        $|\kappa| = \kappa$ となるような順序数のことを基数(cardinal number)という。
    \end{definition*}
    \begin{remark}
        $\kappa$が基数 $\leftrightarrow \forall \alpha < \kappa(\alpha \not\approx \kappa)$ である。任意の $X$ に対して $||X|| = |X|$ だから $|X|$ は基数である。
    \end{remark}
    \begin{remark}
        無限基数は極限順序数である。
    \end{remark}
    \vspace{1ex}
    
    \begin{theorem}
        自然数 $n \in \omega$ は基数。$\omega$ は基数。
    \end{theorem}
    \begin{proof}
        $n+1 \approx n+2 \rightarrow n \approx n+1$ を示す。その後帰納法を使って $n \not\approx n+1$ を示せれば終わり。
    \end{proof}
    
    \begin{theorem}
        $\forall \alpha \exists \kappa (\ \alpha < \kappa \land \kappa$ は基数 $).$
    \end{theorem}
    \begin{proof} \ \\
        $\alpha\geq\omega$ と仮定して良い。$\kappa > \alpha$ が、$\alpha$ より大きい基数であったとしよう。そのとき、$\alpha \not\approx \kappa$ である。$\beta > \alpha$ が $\beta \approx \alpha$ なら $|\beta| = |\alpha| < \kappa = |\kappa|$ だから、$\beta < \kappa$ となるはずである。\\
        つまり、$\kappa \geq \sup \{\beta : \beta \approx \alpha\}$ となるはずである。\\
        この観察を逆に用いる。$f : \alpha \rightarrow \beta$ が全単射なら、$\alpha$ 上の整列順序 $R$ を $xRy \rightarrow f(x) < f(y)$ とすることで、$type(\alpha, R) = \beta$ とできる。\\
        そこで、$W = \{\ R \subseteq \alpha \times \alpha : R$ は $\alpha$ 上の整列順序 $\}$ として、$C = \{\operatorname{type}(\alpha, R) : R \in W\}$ として、$\kappa = \sup C$ とおく。\\
        $\beta < \kappa, \beta \approx \kappa$ のとき、$\beta \in \gamma$ となる $\gamma \in C$ があって、$\kappa \approx \beta \preceq \gamma \approx \alpha \preceq \kappa.$ したがって $\alpha \approx \kappa$ となる。$\kappa \geq \omega$ だから、$\kappa = 1 + \kappa \approx \kappa + 1$ (前者の等号は帰納法によって示せ、後者は足し算の組み合わせ的定義から導ける)。したがって $\alpha \approx \kappa+1$ なので、$\kappa+1 \in C$ で、$\kappa = \sup C \geq \kappa + 1$ となって矛盾する。\\
        故に $\kappa$ は基数である。$\alpha + 1 \in C$ だから $\kappa > \alpha.$
    \end{proof}
    \begin{definition*}
        $\alpha^+$ を $\alpha$ より大きいうち最小の基数とする。
    \end{definition*}
    \begin{remark}
        上の定理の証明において $\alpha^+ = \kappa$ である。
    \end{remark}
    \vspace{1ex}
    \begin{lemma}
        C が基数からなる集合なら $\sup C$ も基数。
    \end{lemma}
    \begin{proof} \ \\
        略。
    \end{proof}
    \begin{definition*} \ \\
        $X$ が有限であるとは、$|X| = n \in \omega$ となることをいう。無限とは有限でないことを指す。\\
        $X$ が可算でであるとは、$|X| \leq \omega$ となることであり、非可算とは可算でないことをいう。\\
        $\alpha^+$ という形の無限基数を後続基数といい、それ以外の無限基数を極限基数という。
    \end{definition*}

    \begin{definition*} \ \\
        $\alpha$ についての超限再帰によって、
        \begin{itemize}
            \item $\aleph_0 = \omega,$
            \item $\aleph_{\alpha+1} = \aleph_\alpha^+,$
            \item $\alpha$ が極限順序数なら $\aleph_\alpha = \sup \{\aleph_\xi : \xi < \alpha\},$
        \end{itemize}
        と定める。$\omega_\alpha = \aleph_\alpha$ とも書く。
    \end{definition*}
    
    \begin{theorem}
        任意の無限基数は $\aleph_\alpha$ という形である。また、$\aleph_\alpha$ が後続基数であることと $\alpha$ が後続順序数であることは同値。
    \end{theorem}
    \begin{proof} \ \\
        帰納法によって証明する。無限基数 $\kappa$ が後続順序数 $\alpha^+$ なら、$\lambda = |\alpha|$ とすれば $\lambda^+ = \kappa$ であり、帰納法の仮定により $\lambda = \aleph_\alpha$ と書ける。$\kappa = \aleph_{\alpha+1}$ となる。\\
        $\kappa$ が極限基数なら、$\kappa = \sup \{\aleph_\alpha < \kappa : \alpha < \kappa\}$ である。実際、$\xi \in \kappa$ のとき、$\xi < \xi^+ < \kappa.$ 帰納法の仮定により $\xi^+ = \aleph_\beta$ となる $\beta$ があって、 $\kappa > \aleph_\beta \geq \beta$ だから $\xi \in \sup \{\aleph_\alpha < \kappa : \alpha < \kappa\}.$ 逆の包含関係は良い。\\
        そこで、$\gamma = \sup \{\alpha < \kappa : \aleph_\alpha < \kappa\}$ とすれば $\kappa = \aleph_\gamma.$ 実際、$\kappa$ は極限順序数だから $\gamma$ は極限順序数となることが示せ、$\aleph_\gamma$ の定義より直ちに従う。
    \end{proof}
    \newpage
    
    \section{基数の演算}
    \begin{definition*} \ \\
        集合 $A, B$ に対して、$^AB = \{\ f \subseteq A \times B : f$ は関数 $ \}$ と定める。基数 $\kappa, \lambda$ に対して
        \begin{itemize}
            \item $\kappa \oplus \lambda = |\kappa \times \{0\} \cup \lambda \times \{1\}|,$
            \item $\kappa \otimes \lambda = |\kappa \times \lambda|,$
            \item $\kappa^\lambda = |^\lambda\kappa|$
        \end{itemize}
        と定める。
    \end{definition*}
    
    \begin{lemma}
        $\kappa$ が無限基数のとき $\kappa \otimes \kappa = \kappa.$
    \end{lemma}
    \begin{proof} \ \\
        帰納法で示す。$\kappa \times \kappa$ 上の辞書式順序を $\lhd$ として、$\kappa \times \kappa$ 上の整列順序 $R$ を\\
        $\langle \alpha, \beta \rangle R \langle \alpha', \beta' \rangle \leftrightarrow \max(\alpha, \beta) < \max(\alpha', \beta') \lor (\max(\alpha, \beta) = \max(\alpha', \beta') \, \land \, \langle \alpha, \beta \rangle \lhd \langle \alpha', \beta' \rangle )$\\
        と定義する。このとき、$\langle \alpha, \beta \rangle \in \kappa \times \kappa$ による切片は $(\max(\alpha, \beta)+1) \times (\max(\alpha, \beta)+1)$ の部分集合だから、\\
        $\operatorname{type}(\kappa \times \kappa, R) = \delta$ について、$\gamma < \delta \rightarrow |\gamma| \leq |(\max(\alpha, \beta) + 1) \times (\max(\alpha, \beta) + 1)| < \kappa.$ 最後の不等号は帰納法の仮定によるものである。\\
        このことから、$|\kappa \times \kappa| \leq \delta \leq \kappa$ が言えた。
    \end{proof}
    
    \begin{theorem} \label{cardinal_operations}
        $\kappa, \lambda$ が、どちらか一方は無限であるような基数のとき $\kappa \oplus \lambda = \kappa \otimes \lambda = \max(\kappa, \lambda).$
    \end{theorem}
    \begin{proof}
        $\max(\kappa, \lambda) \leq \kappa \oplus \lambda \leq \kappa \otimes \lambda \leq \max(\kappa, \lambda) \otimes \max(\kappa, \lambda) = \max(\kappa, \lambda).$
    \end{proof}
    
    \begin{remark}
        基数の演算については結合法則や分配法則、交換法則、そして指数法則が成立する。\\
        基数の演算と順序数の演算は別物である。たとえば、順序数の意味での $\omega^\omega$ は可算順序数だが、基数の意味での $\omega^\omega$ は、$\omega$ より大きい基数である。基数の演算と順序数の演算は自然数については一致する。\\
        $|\mathcal{P}(\kappa)| = 2^\kappa$ である。カントールの定理 (\ref{Cantor}) より $\kappa < 2^\kappa$ なので $\kappa^+ \leq 2^\kappa$ であることがわかる。
    \end{remark}
    
    \begin{definition*}\ \\
        連続体仮説 $\rm(CH)$ とは、言明「$2^{\aleph_0} = \aleph_1$ 」のことである。\\
        一般連続体仮説 $\rm(GCH)$ とは、言明「$\forall \alpha (2^{\aleph_\alpha} = \aleph_{\alpha+1})$ 」のことである。
    \end{definition*}
    \newpage
    
    \section{共終数と正則基数}
    \begin{definition*} \label{cofinal}
       極限順序数 $\alpha$ と関数 $f : \beta \rightarrow \alpha$ について、$f$ が非有界(unbounded)である、あるいは、共終(cofinal)に写すとは、$\forall \xi < \alpha \, \exists \eta < \beta \,(\,f(\eta) > \xi\,)$ となることをいう。言い換えると、$\sup\,\{f(\eta) : \eta < \beta\} = \alpha.$\\
       $\alpha$ の共終数(cofinality) $\operatorname{cf}(\alpha)$ とは、$\alpha$ へ共終に写すような関数が存在するような最小の順序数のことである。
    \end{definition*}
    
    \begin{lemma}
        単調増加な共終に写す関数 $f : \operatorname{cf}(\alpha) \rightarrow \alpha$ が存在する。
    \end{lemma}
    \begin{proof}
        共終に写す関数 $g : \operatorname{cf}(\alpha) \rightarrow \alpha$ を取り、$f$ を超限再帰によって
        $$f(\xi) = \max\,(\,g(\xi), \ \sup \, \{f(\xi') : \xi' < \xi\} + 1\,)$$
        と定めれば良い。この定義は、$\xi < \operatorname{cf}(\alpha)$ のとき $f\restriction\xi : \xi \rightarrow \alpha$ は共終数の定義より有界であることと、$\alpha$ が極限順序数であることから、単調増加かつ共終に写す関数 $f : \operatorname{cf}(\alpha) \rightarrow \alpha$ を定義する。
    \end{proof}
    \begin{lemma} \label{cf_eq_cf}
        $\alpha, \beta$ が極限順序数で、単調増加な共終に写す関数 $f : \alpha \rightarrow \beta$ が存在するなら、$\operatorname{cf}(\alpha) = \operatorname{cf}(\beta)$ である。
    \end{lemma}
    \begin{proof}
        共終に写す単調増加な関数 $g : \operatorname{cf}(\alpha) \rightarrow \alpha, \ h : \operatorname{cf}(\beta) \rightarrow \beta$ を取る。\\
        $f\circ g : \operatorname{cf}(\alpha) \rightarrow \beta$ は共終に写す。
        実際、$\eta < \beta$ に対し、$f(\xi) > \eta$ となる $\xi < \alpha$ を取り、$g(\xi') > \xi$ となる $\xi' < \operatorname{cf}(\alpha)$ を取れば、単調増加性より $f(g(\xi')) > f(\xi) > \eta.$ よって $\operatorname{cf}(\beta) \leq \operatorname{cf}(\alpha).$\\
        $F : \operatorname{cf}(\beta) \rightarrow \alpha$ を、$\eta < \operatorname{cf}(\beta)$ に対して、$h(\eta) < f(\xi)$ となる最小の $\xi < \alpha$ を $F(\eta)$ とすれば、これは共終に写す。実際、$\xi < \alpha$ に対して、$f(\xi) < h(\eta)$ となる $\eta < \operatorname{cf}(\beta)$ がある。定義より $h(\eta) < f(F(\eta))$ だから $f(\xi) < f(F(\eta))$ より、単調増加であるから $\xi < F(\eta).$ よって $\operatorname{cf}(\beta) \geq \operatorname{cf}(\alpha).$
    \end{proof}

    \begin{theorem}
        $\alpha$ が極限順序数のとき $\operatorname{cf}(\operatorname{cf}(\alpha)) = \operatorname{cf(\alpha)}.$
    \end{theorem}
    
    \begin{definition*}
        $\operatorname{cf}(\alpha) = \alpha$ となる極限順序数を正則である、という。正則な順序数は無限基数である。正則でない基数を特異基数という。
    \end{definition*}
    \begin{remark}
        $\gamma$ を極限順序数として、各 $\xi < \gamma$ に対して、単調増加な順序数列 $\langle \alpha_\xi : \xi<\gamma \rangle$ が定まっているとする。このとき、$\displaystyle \lim_{\xi \to \gamma} \alpha_\xi = \sup \{ \alpha_\xi : \xi < \gamma\}$ と定義することができる。\\
        $\operatorname{cf}(\alpha)$ は、$\displaystyle \lim_{\xi \to \gamma} \alpha_\xi = \alpha$ となるような順序数列 $\langle \alpha_\xi : \xi<\gamma \rangle$ が存在するような $\gamma$ のうち最小のものである。\\
        正則基数とは、自身よりも小さい基数による極限として表現できないような基数のことである。
    \end{remark}
    
    \vspace{0.5ex}
    
    選択公理のもとでは、後続基数は正則である。以下ではそれを示そう。
    \begin{lemma} (AC)
        $f : A \rightarrow B$ が全射なら、単射 $g : B \rightarrow A$ が存在する。
    \end{lemma}
    \begin{lemma} (AC) \label{union_kappa_kappa}
        $\kappa$ を無限基数として、各 $\alpha < \kappa$ に対して集合 $X_\alpha$ が定まっており、$|X_\alpha| \leq \kappa$ であるとする。 $\displaystyle \left| \bigcup_{\alpha < \kappa} X_\alpha \right| \leq \kappa$ である。
    \end{lemma}
    \begin{proof}
        各 $\alpha < \kappa$ に対して $f_\alpha : \kappa \rightarrow X_\alpha$ を選択する。$\displaystyle \kappa \times \kappa \rightarrow \bigcup_{\alpha < \kappa} X_\alpha$ を $\langle \alpha, \beta \rangle \mapsto f_\alpha(\beta)$ と定めると全射である。したがって補題より $\displaystyle \left| \bigcup_{\alpha < \kappa} X_\alpha \right| \leq \kappa \otimes \kappa = \kappa.$
    \end{proof}
    
    \begin{theorem} (AC)
        任意の後続基数 $\kappa^+$ は正則である。
    \end{theorem}
    \begin{proof}
        正則でないとすると、共終に写す $f : \kappa \rightarrow \kappa^+$ が存在する。$\alpha < \kappa$ に対し、$X_\alpha = \{\xi < \kappa^+ : \xi < f(\alpha)\}$ とすると、$\displaystyle \bigcup_{\alpha < \kappa} X_\alpha = \kappa^+.$\\
        しかし、$f(\alpha) < \kappa$ より $|X_\alpha| < \kappa^+$ なので $|X_\alpha| \leq \kappa.$ (\ref{union_kappa_kappa}) より $|X_\alpha| \leq \kappa$ であるとする。 $\displaystyle \left| \bigcup_{\alpha < \kappa} X_\alpha \right| \leq \kappa$ となって、矛盾。
    \end{proof}
    
    \vspace{1.0ex}
    これで、選択公理のもとでは後続基数は正則であることがわかった。正則な極限基数としては $\omega$ があるが、これよりも大きい正則な極限基数はあるのだろうかという疑問が湧く。
    
    \begin{definition*}
        正則な非可算極限基数を、弱到達不能基数(weakly inaccessible cardinal)という。
    \end{definition*}
    \begin{proposition}
        $\aleph_\alpha$ が弱到達不能基数ならば $\alpha = \aleph_\alpha$ である。
    \end{proposition}
    \begin{proof}
        $\alpha \rightarrow \aleph_\alpha, \ \xi \mapsto \aleph_\xi$ は共終に写すので (\ref{cf_eq_cf}) より $\operatorname{cf}(\alpha) = \operatorname{cf}(\aleph_\alpha) = \aleph_\alpha.$ したがって、$\alpha \geq \operatorname{cf}(\alpha) = \aleph_\alpha.$ 逆の不等号は自明。
    \end{proof}
    \begin{remark}
        この命題は、弱到達不能基数が非常に大きいことを示している。$\aleph_\alpha = \alpha$ となる最小の基数を構成してみよう。$\sigma_0 = \aleph_0, \ \sigma_{n+1} = \aleph_{\sigma_n}$ として $\sigma = \sup \, \{\sigma_n : n \in \omega\}$ とすれば、これは最小のものである。\\
        まず、$\sigma$ は極限順序数である。\\
        したがって、$\aleph_\sigma = \sup \, \{\aleph_\xi : \xi < \sigma\} = \sup \, \{\aleph_{\sigma_n} : n \in \omega\} = \sup \{\sigma_{n+1} : n \in \omega\} = \sigma.$ ただし、二番目の等号は、$\xi \in \sigma \leftrightarrow \exists n \in \omega \, (\xi < \sigma_n)$ を使った。\\
        次に、$\aleph_\alpha = \alpha$ であるとき、$\sigma_0 \leq \alpha$ であり、$\sigma_n \leq \alpha$ であるなら $\sigma_{n+1} = \aleph_{\sigma_n} \leq \aleph_{\alpha} = \alpha$ なので、帰納法とあわせて $\sigma \leq \alpha = \aleph_\alpha.$\\
        しかし、$\sigma$ は$\omega$-長の単調増加な順序数列の極限になっているから、$\operatorname{cf}(\sigma) = \omega$ であり、弱到達不能ではない。
    \end{remark}
    
    \vspace{1.0ex}
    
    \begin{definition*}
        基数 $\kappa$ が強極限(strong limit)であるとは、任意の基数 $\lambda < \kappa$ に対して $2^\lambda < \kappa$ となることをいう。
    \end{definition*}
    \begin{remark}
        $\kappa$ を無限基数とする。$\kappa_0 = \kappa, \ \kappa_{n+1} = 2^{\kappa_{n}}$ として $\kappa_{\omega} = \sup \, \{\kappa_n : n \in \omega\}$ は $\kappa$ より大きいうち最小の強極限である。$\operatorname{cf}(\kappa_\omega) = \omega.$
    \end{remark}
    \begin{definition*}
        弱到達不能かつ強極限であることを、(強)到達不能(inaccessible)であるという。
    \end{definition*}
    
    \newpage
    \section{基数の冪乗}
    \begin{theorem} \label{Konigs_lemma} (Kőnig) \\
        $\kappa$ を無限基数とする。$\lambda \geq \operatorname{cf}(\kappa)$ なら $\kappa^\lambda > \kappa$ である。
    \end{theorem}
    \begin{proof}
        $\lambda = \operatorname{cf}(\kappa)$ として良い。\\
        $F : \kappa \rightarrow {^\lambda\kappa}$ が全射でないことを示そう。$F$ による像に入っていないような $f \in {^\lambda\kappa}$ を構成したい。\\
        $f$ が $F$ による像に入っていない ことと、$\forall \xi \in \kappa \ (F(\xi) \neq f)$ となることは同値である。また、これは $\forall \xi \in \kappa \ \exists \eta \in \lambda \ (\,F(\xi)(\eta) \neq f(\eta)\,)$ と同値である。\\
        そこで、$\eta \in \lambda$ に対して、$X_\eta = \{\xi \in \kappa : F(\xi)(\eta) \neq f(\eta)\}$ とおく。上の条件は $\bigcup_{\eta < \lambda} X_\eta = \kappa$ と同値。\\
        いま、$h : \lambda \rightarrow \kappa$ を共終に写す写像であるとする。例えば $X_\eta = \{\xi < \kappa : \xi < h(\eta)\}$ なら $\displaystyle \bigcup_{\eta < \lambda} X_\eta = \kappa.$\\
        そこで、$f(\eta) = \min \, (\kappa - \{F(\xi)(\eta) : \xi < h(\eta)\})$ とすることができる。これは $\kappa$ が基数であり $h(\eta) < \kappa$ であることから $\min$ の中にある集合は空ではなく、well-defined である。\\
        このように定義すれば、$X_\eta = \{\xi < \kappa : \xi < h(\eta)\}$ となって、$f$ は $F$ の像に入っていない。
    \end{proof}
    
    \begin{theorem} \label{cardinal_power} (AC)\\
        $\kappa, \lambda$ を基数で、少なくとも片方が無限であり、どちらも $2$ 以上とする。
        \begin{enumerate}[(i)]
            \item $\kappa \leq \lambda$ なら $\kappa^\lambda = 2^\lambda.$ さらに GCH を仮定すれば $\kappa^\lambda = \lambda^+.$
            \item $\kappa > \lambda \geq \operatorname{cf}(\kappa)$ なら $\kappa < \kappa^\lambda \leq 2^\kappa.$ さらに GCH を仮定すれば $\kappa^\lambda = \kappa^+.$
            \item GCH を仮定すれば $\operatorname{cf}(\kappa) > \lambda$ なら $\kappa^\lambda = \kappa.$
        \end{enumerate}
    \end{theorem}
    \begin{proof} \ \\
        (i): $2^\lambda \leq \kappa^\lambda \leq \lambda^\lambda.$ また、(\ref{cardinal_operations}) より $^\lambda\lambda \subseteq \mathcal{P}(\lambda \times \lambda) \simeq \mathcal{P}(\lambda) \simeq 2^\lambda$ だから、$\lambda^\lambda = 2^\lambda$ であり、示された。\\
        (ii): $\kappa < \kappa^\lambda$ は (\ref{Konigs_lemma}) で示されている。$\kappa^\lambda \leq \kappa^\kappa = 2^\kappa.$\\
        (iii): $\operatorname{cf}(\kappa) > \lambda$ のとき、$\displaystyle ^\lambda \kappa = \bigcup_{\alpha < \kappa} {^\lambda \alpha}$ である。(iii) が $\kappa$ 未満の基数で成立すると仮定する。$\alpha < \kappa$ のとき、$|\alpha| < \kappa$ だから、$^\lambda\alpha \simeq |\alpha|^\lambda \leq \max(|\alpha|, \lambda)^+ \leq \kappa.$ (\ref{Konigs_lemma}) より、$|{^\lambda \kappa}| \leq \kappa.$ よって示された。
    \end{proof}
    
    \section{参考文献}
    \begin{enumerate}[]
        \item Jech, T. [2006] {\it Set Theory} (Springer, Berlin)
        \item Kunen, K. [1992] {\it Set Theory} (North Holland, Netherlands)
    \end{enumerate}

\end{document}