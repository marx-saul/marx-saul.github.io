\documentclass{article}

\usepackage{amsmath, amssymb, amsthm, amscd}
\usepackage{enumerate}

\theoremstyle{theorem}
\newtheorem*{definition*}{定義}
\newtheorem{theorem}{Theorem}[section]
\newtheorem{lemma}[theorem]{Lemma}
\newtheorem{corollary}[theorem]{Corollary}
\newtheorem{remark}{Remark}[section]
\newtheorem{example}{Example}[section]
\renewcommand\proofname{\rm (Proof)}

\parindent = 0pt

\begin{document}
    \section{Hilbert Polynomial}
    Let $X$ be a closed subscheme of a projective space $\mathbb{P}^N_k$ over a filed $k$. It is well-known that the cohomology groups $H^i(X, \mathcal{F})$ of a coherent sheaf $\mathcal F$ on $X$ is a finite-dimensional vector space. Therefore we can define the Euler characteristic $\displaystyle \chi(\mathcal F) := \sum_{i = 0}^{\infty} (-1)^i \dim_k H^i(X, \mathcal{F}).$
    
    Fix a very ample invertible sheaf $\mathcal{O}_X(1)$. We will show that $\phi_{\mathcal F}(n) = \chi(\mathcal F(n))$ is a polynomial in $n \in \mathbb N$, which is called the Hilbert polynomial.
    \newline
    
    First of all, one can reduce to the case where $X =\mathbb{P}^N_k.$ Indeed, letting $\iota : X \rightarrow \mathbb{P}^N_k$ be the closed immersion, we have $H^i(X, \mathcal{F}(n)) = H^i(\mathbb P^N_k, (\iota_*\mathcal{F}(n))) = H^i(\mathbb P^N_k, (\iota_*\mathcal{F})(n))$ and the sheaf $\iota_*\mathcal F$ is coherent on $\mathbb{P}^N_k$. The last equation comes from the projection formula. In what follows in this section we let $X =\mathbb{P}^N_k.$
    
    Let $S = k[t_0, \cdots, t_N]$. Induction on $N$. If $N = 0$, then $\mathbb{P}^N_k = \operatorname{Spec} k$ and it is obvious. Consider the morphism $\mathcal F(-1) \xrightarrow{\cdot t_N} \mathcal F$, and let $\mathcal K$ and $\mathcal C$ be its kernel and cokernel. There is an exact sequence
    $$ 0 \rightarrow \mathcal K(n) \rightarrow \mathcal F(n-1) \rightarrow \mathcal F(n) \rightarrow \mathcal C(n) \rightarrow 0.$$
    Since $\mathcal K$ and $\mathcal C$ vanishes by the multiplication by $t_N$, it can be considered as a coherent morphism on $H = V_+(t_N) \simeq \mathbb{P}^{N-1}_k$ (precisely, letting $\iota : H \rightarrow \mathbb{P}^{N}_k$ be the closed immersion, there is a coherent sheaf $\mathcal K'$ and $\mathcal C'$ on $H$ such that $\iota_*\mathcal K' = \mathcal K$ and $\iota_*\mathcal C' = \mathcal C$).
    
    Take the Euler characteristic we get $\chi(\mathcal F(n)) - \chi(\mathcal F(n-1)) = \chi(\mathcal C(n)) - \chi(\mathcal K(n)).$
    
    The right-hand side is a polynomial in $n$ by inductive hypothesis. As $\displaystyle \sum_{m = 0}^{n} m^i$ is written in a polynomial in $n$ of degree $i+1$, we get the result. $\Box$
    \newline
    
    By the Serre's theorem, $\chi(\mathcal F(n)) = \dim_k H^0(X, \mathcal F(n))$ for sufficiently large $n.$ The right-hand side is called the Hilbert function. Letting $M = \Gamma_*(\mathcal F)$, we have $\dim_k H^0(X, \mathcal F(n)) = \dim_k M_n$. Hence it is the Hilbert function of the graded $S$-module $M$.
    
    By the dimension theory, we have $\deg \phi_{\mathcal F} \leq \dim \operatorname {Supp} \mathcal F.$ When $\mathcal F = \mathcal \iota_*O_Y$ is the structure sheaf of $Y$ pushed forward into $X$ via the closed immersion $\iota: Y \rightarrow X$,  then the equality holds.
    
    In order to show this, we need following two theorems:
    
    \begin{theorem} [{[}Mat 13.7{]}] \label{graded_ring_prime}
        Let $\displaystyle A = \bigoplus_{n \geq 0} A_n$ be a graded Noetherian ring. Then:
        \begin{enumerate}[(i)]
            \item for a homogeneous ideal $I$, its prime divisors are also homogeneous;
            \item for a homogeneous ideal $P$ of height r, there is a prime chain $P = P_0 \supsetneq P_1 \supsetneq \cdots \supsetneq P_r$ consisting of only homogeneous prime ideals.
        \end{enumerate}
    \end{theorem}
    
    
    \begin{theorem} [{[}Mat 13.8{]}] \label{graded_ring_dimension}
        Let $k$ be a field and $\displaystyle R = k[\xi_1, \cdots, \xi_n]$ be a graded $k$-algebra generated by homogeneous elements $\xi_i$ of degree 1. Let $\mathfrak m = R_+ = (\xi_1, \cdots, \xi_n)$ and let $h(n) = \dim_k R_n$ be the Hilbert function of $R$. Then we have $\dim R = \operatorname{ht} \, \mathfrak m = \deg h + 1.$
    \end{theorem}
    
    Using these we get $\dim Y = \dim (\operatorname{Proj} S/I) = \operatorname{ht} \ (t_0, \cdots, t_n) (S/I) - 1 = \deg \, \phi_{\mathcal O_Y}.$
    
    For a coherent sheaf $\mathcal F$ with support $Y$, we have an exact sequence 
    $$\displaystyle 0 \rightarrow \mathcal K \rightarrow \bigoplus_{i = 1}^{r} \mathcal O_Y(q_i) \rightarrow \mathcal F \rightarrow 0 $$
    
    for some $q_i$. Tensoring with $\mathcal O_X(n)$ for a sufficiently large and take the Euler charactersitic, we get $\displaystyle \phi_{\mathcal F}(n) \leq \sum_{i=1}^{r} \phi_{\mathcal O_Y}(n + q_i).$ This shows $\deg \phi_{\mathcal F} \leq \dim Y.$
    
    \section{Flatness}
    
    Flatness is a purely algebraic notion, but it plays an important role in algebraic geometry.
    
    \begin{theorem} [{[}Har III.9{]}] \label{flat_hilbert_polynomial}
        Let $X$ be a projective scheme over a connected Noetherian scheme $S$. For a coherent sheaf $\mathcal F$ on $X$, if $\mathcal F$ is flat over $S$, then the Hilbert polynomial of $\mathcal F_s$ on the fiber $X_s \subseteq \mathbb P^n_{k(s)}$ is independent of $s \in S$. Furthermore if $S$ is integral, then the converse is true.
    \end{theorem}
    
    By this theorem we can define the Hilbert polynomial for a coherent sheaf $\mathcal F$ flat over $S$.
    
    The aim of this chapter is to show the following theorem:
    
    \begin{theorem} [Flatteness Stratification {[}Mum Lecture 8{]}] \label{flattening_stratification}
        Let $X$ be a projective scheme over a Noetherian scheme $S$ and let $\mathcal F$ be a coherent sheaf on $X$. There are finitely many disjoint locally closed subschemes $S_1, \cdots, S_m$ of $S$ such that $S = \bigcup S_i$ with the following property:
        
        for a morphism $g: T \rightarrow S$ with $T$ Noetherian, let $g_T: X_T = X\times_S T \rightarrow T$ be the base change; then $g_T^*\mathcal F$ on $X_T$ is flat over $T$ if and only if $g: T \rightarrow S$ factors through $\coprod S_i \rightarrow S.$
    \end{theorem}
    
    We follow the method of {[Mum]}. Note that we can reduce to the case where $X = \mathbb P^n_S$. In order to prove this, we need several facts.
    
    For a point $s \!\in\! S$, we let $\mathcal F_s$ be the pull-back of $\mathcal F$ on $\mathbb P^n_{k(s)}.$ By the Serre's theorem, for any $s \!\in\! S$, there is an $m_0 \!>\! 0$, depending on $s$, such that $H^i(\mathbb P^n_{k(s)}, \mathcal F_s(m)) = 0$ for all $m \!\geq\! m_0$ and $i \!>\! 0$. First thing is to take a uniform $m_0$, that is, an $m_0$ independent of $s \in S$.
    
    Since $S$ is quasi-compact, we only have to show that an $m_0$ for $s$ propagates for points around $s$. When $\mathcal F$ is flat over $S$, this works:
    \newpage
    
    \begin{theorem} [Upper Semicontinuity {[}Har III.12.8{]}] \label{upper_semicontinuity}
        If $\mathcal F$ is flat over $S$, then the map
        $$h^i(s) = \dim_{k(s)} H^i(X_s, \mathcal F_s)$$ is upper semicontinuous, i.e., for any $s \in S$, there is an open neighborhood $U$ of $s$ such that $h^i(s') \leq h^i(s)$ for $s' \in U.$ 
    \end{theorem}
    
    For the general case, let us show the following lemma:
    
    \begin{lemma} \label{flat_decomposition}
        Let $f:X \rightarrow S$ be a morphism of finite type between Noetherian schemes and let $\mathcal F$ be a coherent sheaf on $X$. There exist finitely many irreducible locally closed subsets $Y_1, \cdots, Y_r$ of $S$ that cover $S$ and that if we endow $Y_i$ with the reduced subscheme structure, then $\iota^*\mathcal F$ is flat over $Y_i$, where $\iota : f^{-1}(Y_i) \rightarrow X$ is the base change of the canonical immersion $Y_i \rightarrow S$.
    \end{lemma}
    
    To show this, we need some algebraic result:
    
    \begin{lemma} [{[}Mat 7.9{]}] \label{flat_exact_sequence}
        Let $$0 \rightarrow M' \rightarrow M \rightarrow M'' \rightarrow 0$$ be an exact sequence of $A$-modules. If $M'$ and $M''$ are flat, then so is $M$ flat.
    \end{lemma}
    
    \begin{lemma} \label{locally_free}
        Let $A$ be a Noetherian integral domain, let $B$ be a finitely generated $A$-algebra and let $M$ be a finite $B$-module. There is a nonzero $f \in A$ such that $M_f$ is free over $A_f$.
    \end{lemma}
    \begin{proof}
        First, there is a sequence of sub-$B$-modules
        $$ 0 = M_0 \subseteq M_1 \subseteq \cdots \subseteq M_n = M$$
        such that $M_i/M_{i-1} \simeq B/P_i$ for some prime ideal $P_i$ of $B$ {[}Mat 6.4{]}.
        
        Now we may suppose that $M = B$ and $B$ is an integral domain. Let $K$ and $L$ be the field of fractions of $A$ and $B$, resp. Let $B = A[x_1, \cdots x_n]$ and $B' = K[x_1, \cdots, x_n] \subseteq L.$ Apply Noether's normalization lemmato $B'$, there exist  $z_1, \cdots, z_r \in B$ algebraically independent over $K$ such that $B'$ is integral over $K[z_1, \cdots, z_r]$ ($r = {\rm tr.deg}_K L$).
        
        Each $x_i$ has an algebraic relation by a monic polynomial with coefficients in $K[z_1, \cdots, z_r]$. If we let $f$ be the product of all the denominators of these coefficients, each $x_i$ is integral over $A_f[z_1, \cdots, z_r].$
        
        Therefore, $B_f = A_f[x_1, \cdots, x_n]$ is integral over $A' := A_f[z_1, \cdots, z_r]$. It is also finitely generated, so $B_f$ is a finite $A'$-module. Write $B_f = A'y_1 + \cdots +A'y_m$. Let $K' = K(z_1, \cdots, z_r)$ be the field of fractions of $A'$ and let $y_1, \cdots, y_s$ be the basis for the module $K'y_1 + \cdots + K'y_m.$
        
        Then there is an exact sequence of $A_f$-modules
        $$0 \rightarrow A'^s \rightarrow B_f \rightarrow D \rightarrow 0$$
        with $D$ annihilated by some nonzero element $g \in A'.$
        
        Now we use the same method to show that $D_f'$ is free over $A_f'$ for some $f'$. Then the rings that appear has transcendental degree $< \! r$. By induction on $r$, we get some $f'$ and then $B_{ff'}$ is free over $A_{ff'}$ (note that $A'$ is free over $A_f$).
        
        It now suffices to show the case where $r = 0$. $B$ is integral over $A$ and $L/K$ is a finite field extension. Let $d_i = [K(x_1, \cdots, x_i) \!:\! K(x_1, \cdots, x_{i-1})]$. Then $\{ \prod_{e} x_i^{e_i} : 0 \! \leq \! e_i \! < \! d_i \}$ is a basis for $B$ over $A$.
    \end{proof}
    
    Now we can prove (\ref{flat_decomposition}):
    
    \begin{proof}
        We endow each irreducible components $Z$ of $S$ the reduced induced closed subscheme structure. It suffices to show that the lemma holds for each $Z$, so we may assume that $S$ is integral.
        
        By (\ref{locally_free}), there is a nonempty open set $V$ of $Y$ such that $\mathcal F|_{f^{-1}(V)}$ is flat over $f^{-1}(V)$.
        
        Now by the Noetherian induction of closed subsets we get the lemma.
    \end{proof}
    
    Now by (\ref{flat_decomposition}) and by the flat case, we have a unifrom $m_0 > 0$ depending only on $\mathcal F$ such that $H^i(\mathbb P^n_{k(s)}, \mathcal F_s(m)) = 0$ for every $s \in S$, $m \geq m_0$ and $i > 0$.
    
    By (\ref{flat_hilbert_polynomial}) it suffices to see the Hilbert polynomial of $\mathcal F$. In order to show (\ref{flattening_stratification}) we may assume that $S$ is connected. For $m \geq m_0,$ we have $\phi_{\mathcal F}(m) = \dim H^0(\mathbb P^n_{k(s)}, \mathcal F_s(m))$ for some $s \in S.$
    
    Let $p : \mathbb P^n_S \rightarrow S$ be the structure morphism and let $\mathcal E_m = p_* \mathcal F(m).$ Then we have a useful relation between $\mathcal E_m \otimes k(s)$ and $H^0(\mathbb P^n_{k(s)}, \mathcal F_s).$
    
    \begin{theorem} [Grauert's theorem {[Har III.12.9]}] \label{Grauert_theorem}
        Let $S$ be an integral Noetherian scheme, let $f : X \rightarrow S$ be a projective morphism and let $\mathcal F$ be a coherent sheaf on $X$ flat over $S$. If the map $s \rightarrow \dim_{k(s)} H^i(X_s, \mathcal F_s)$ is constant for $s \in S$, then $R^if_*(\mathcal F) \otimes k(s) \rightarrow H^i(X_s, \mathcal F_s)$ is an isomorphism for every $s \in S$.
    \end{theorem}
    
    Let us fix an irreducible locally closed subscheme $Y$ of $S$, and consider the following diagram:
    \[
      \begin{CD}
         \mathbb  P^n_Y @>{h}>> \mathbb  P^n_S \\
      @V{q}VV    @VV{p}V \\
         Y   @>{g}>>  S
      \end{CD}
    \]
    \newline
    
    Suppose that $h^*\mathcal F$ is flat over $Y$. The Hilbert polynomial of $(h^* \mathcal F)_s = \mathcal F_s$ for $s \in Y$, which is equal to $\dim H^0(\mathbb P^n_{k(s)}, \mathcal F(m)_s)$ for $m \geq m_0$, is independent of $s$. By Grauert's theorem we have the isomorphism $q_*h^*\mathcal F(m) \otimes k(s) \rightarrow H^i(\mathbb P^n_{k(s)}, \mathcal F(m)_s)$.
    
    There is a canonical morphism $g^*\mathcal E_m = g^*p_* \mathcal F(m) \rightarrow q_*h^*\mathcal F(m).$ This is an isomorphism of sufficiently large $m$ {[Mum Lecture 7, 3$^\circ$, (i)]}.
    
    Thus for sufficiently large $m$, we have an isomorphism $\mathcal E_m \otimes k(s) = g^*\mathcal E_m \otimes k(s) \rightarrow H^0(\mathbb P^n_{k(s)}, \mathcal F(m)_s)$ for $s \in S$. Since one can cover $S$ with finite number of $Y$'s, $\mathcal E_m \otimes k(s) \rightarrow H^0(\mathbb P^n_{k(s)}, \mathcal F(m)_s)$ is an isomorphism for every $s \in S$ and $m \geq m_1$ for some $m_1 \geq m_0$.
    \newpage
    
    \begin{theorem} [Cohomology and Base Change {[Har III.12.11]}] \label{cohomology_base_change}
        Let $f : X \rightarrow S$ be a projective morphism of Noetherian schemes and let $\mathcal F$ be a coherent sheaf on $X$ flat over $S$. For a point $s \in S$ we let $\psi^i_s : R^if_*(\mathcal F) \otimes k(s) \rightarrow H^i(X_s, \mathcal F_s)$ be the canonical morphism. Suppose that $\psi^i_s$ is surjective. Then:
        \begin{enumerate}[(i)]
            \item $\psi^i_s$ is an isomorphism, and the same holds for $s'$ around $s$;
            \item $\psi^{i-1}_s$ is surjective if and only if $R^if_*(\mathcal F)$ is locally free in a neighborhood of $s$.
        \end{enumerate}
    \end{theorem}
    
    \begin{theorem} \label{projective_flat}
        Let $S$ be a Noetherian scheme, let $p : \mathbb P^n_S \rightarrow S$ be the projective space and let $\mathcal F$ be a coherent sheaf on $\mathbb P^n_S$. Then $\mathcal F$ is flat over $S$ if and only if $p_* \mathcal F(m)$ is finite locally free for sufficiently large $m$.
    \end{theorem}
    \begin{proof}
        Take an $M > 0$ such that $R^if_*(\mathcal F(m)) = 0$ for $i > 0$ and $m \geq M$. Then  (\ref{cohomology_base_change}) implies that $p_*\mathcal F(m) \otimes k(s) \rightarrow H^0(X_s, \mathcal F(m)_s)$ is an isomorphism.
        
        If $\mathcal F$ is flat over $S$, then the Hilbert polynomial of $\mathcal F_s$ is locally constant for $s \in S$ (\ref{flat_hilbert_polynomial}). For sufficiently large $m$, the Hilbert polynomial is equal to $\dim_{k(s)} H^0(X_s, \mathcal F(m)_s)$, so $p_* \mathcal F(m)$ is finite locally free [Stacks 05P2].
        
        Conversely suppose that $p_*\mathcal F(m)$ is locally free for $m \geq m_0$.  We may assume that $S = \operatorname{Spec} R$ is an affine spectrum of a Noetherian ring $R$. $H^0(S, p_*\mathcal F(m)) = H^0(X, \mathcal F(m))$ is a finite free $A$-module. Let $\displaystyle M = \bigoplus_{m \geq m_0} H^0(X, \mathcal F(m)).$ Then $\widetilde{M} \simeq \mathcal F$ is flat over $S$, as easily seen.
    \end{proof}
    
    Now we are ready to show (\ref{flattening_stratification}).
    
    We have an $m_1 > 0$ such that $\mathcal E_m \otimes k(s) \rightarrow H^0(\mathbb P^n_{k(s)}, \mathcal F(m)_s)$ is an isomorphism and $H^i(\mathbb P^n_{k(s)}, \mathcal F(m)_s) = 0$ for every $i > 0, s \in S$ and $m \geq m_1$.
    
    Let $g : T \rightarrow S$ be a morphism of Noetherian schemes, and consider the following diagram:
    \[
      \begin{CD}
         \mathbb  P^n_T @>{h}>> \mathbb  P^n_S \\
      @V{q}VV    @VV{p}V \\
         T   @>{g}>>  S
      \end{CD}
    \]
    \newline
    
    (*): Let $\mathcal F_T = h^*\mathcal F$. For $m \geq m_1$, we have a canonical isomorphism $g^* \mathcal E_m \rightarrow q_*\mathcal F_T(m)$ {[Mum Lecture 7, 3$^\circ$, Corollary 2]}. If $\mathcal F_T$ is flat over $T$, then $g^* \mathcal E_m$ is finite locally free for $m \geq m_1$ by using (\ref{cohomology_base_change}) and the proof in (\ref{projective_flat}). Conversely if $g^* \mathcal E_m$ is finite locally free for all $m \geq m_1$, then $\mathcal F_T$ is flat over $T$ by (\ref{projective_flat}). So let us investigate when $g^*\mathcal E$ is finite locally free.
    \newline
    
    Next we consider the flattening stratification in the case $X = S$. Let $\mathcal E$ be a coherent sheaf on $S$ and $h(s) = \dim_{k(s)} \mathcal E \otimes_{\mathcal O_S} k(s)$. For any $s \in S$, let $m = h(s)$; there is an open neighborhood $U$ of $s$ and an exact sequence
    $$\mathcal O_U^n \xrightarrow{\phi} \mathcal O_U^m \xrightarrow{\psi} \mathcal E \rightarrow 0$$
    by Nakayama's lemma. We let $U_s$ be such a $U$. The set $Z_m = \{ s \in S : h(s) = m \}$ is a locally closed subset of $S$, since $Z_m \cap U_s$ is the set of loci where $\phi$ vanish. We let $Y_s = Z_m \cap U_s$. We endow $Y_s$ with the closed subscheme structure of $U_s$, defined by the ideal generated by all components of the matrix $(a_{ij})$ defining $\phi$.
    
    We claim the following property:
    \newline
    
    (**) for any morphism $h : T \rightarrow U_s$ of Noetherian schemes, $h^*\mathcal E$ is locally free of rank $m$ if and only if $h$ factors through $Y_s$.
    \newline
    
    Indeed, $h$ factors through $Y_s$ if and only if all $h^{\sharp}(a_{ij})$ vanish, i.e., $h^*\phi = 0$.
    
    The sequence
    $$\mathcal O_T^n \xrightarrow{h^*\phi} \mathcal O_T^m \xrightarrow{h^*\psi} h^*\mathcal E \rightarrow 0$$
    is exact, so it is equivalent to saying that $h^*\psi$ is isomorphism; hence $h^*\mathcal E$ is locally free of rank $m$.
    
    Conversely if $h^*\mathcal E$ is locally free of rank $m$, let $\mathcal K$ be the kernel of $h^*\phi$ and we have an exact sequence
    $$ 0 = \operatorname{Tor}_1(h^*\mathcal E, k(t)) \rightarrow \mathcal E \otimes k(t) \rightarrow k(t)^m \rightarrow h^*\mathcal E \otimes k(t) \rightarrow 0.$$
    for $t \in T$. Since the last term is a vector space over $k(t)$ of dimension $m$, we get $\mathcal K \otimes k(t) = 0$. By Nakayama's lemma we get $\mathcal K = 0$. $h^*\psi$ is an isomorphism and $h^*\phi = 0$. $h$ factors through $Y_s$. $\Box$
    \newline
    
    Now the property (**) characterizes the locally closed subscheme $Y_s$ in $Z_m \cap U_s$. That is, for two points $s_1, s_2 \in Z_m$, two locally closed subschemes $Y_{s_1}$ and $Y_{s_2}$ restricted on the open set $U_{s_1} \cap U_{s_2}$ are equal. Thus one can glue to get a locally closed subscheme structure on $Z_m$.  The property (*) gives that for any morphism $h : T \rightarrow S$ of Noetherian schemes and a coherent sheaf $\mathcal E$ on $T$, $g^* \mathcal E$ is finite locally free (of rank $m$) on $T$ if and only if $g : T\rightarrow S$ factors through $\displaystyle \coprod_j Z_j \rightarrow S$ ($Z_m \rightarrow S$). This is the flattening stratification of $\mathcal E$.
    \newline
    
    We can finally prove (\ref{flattening_stratification}). Take $Y_1, \cdots, Y_r$ of $S$ as in (\ref{flat_decomposition}) and flattening stratification $Z^m_j$ of $\mathcal E_m$ for $m \geq m_1$. Let $P_i$ be the Hilbert polynomial of $\mathcal F$ over $Y_i$. We can take the union of two $Y_i$'s that having the same Hilbert polynomials (\ref{flat_hilbert_polynomial}) to assume that $P_i$'s are pairwise distinct.
    
    We claim that set-theoretically $\displaystyle Y_i = \bigcap_{m = m_1}^{m_1 + n} Z^m_{P_i(m)}$. For an $s \in S$ let $P_j$ be the Hilbert polynomial of $\mathcal F_s$; $s$ is contained in the right-hand side if and only if $P_j(m) = \dim_{k(s)} H^0(\mathbb P^n_{k(s)}, \mathcal F_s) = \dim_{k(s)} \mathcal E_m \otimes k(s) = P_i(m)$ for $m = m_1, \cdots, m_1 + n$. Hilbert polynomials have the degree $\leq n$, so we have $P_i = P_j$ and that $s \in Y_i.$ The converse inclusion is obvious.
    
    Now endow the locally closed subsets $Y_i$ (which has the reduced subscheme structure) with the intersection structure of $\displaystyle \bigcap_{m = m_1}^{\infty} Z^m_{P_i(m)}$, and denote this by $S_i$. This is possible since it is the limit of locally closed subschemes with a fixed underlying space.
    
    These $S_i$'s are the flattening stratification of $\mathcal F$. If $T \rightarrow S$ is a morphism of Noetherian schemes, let us consider the following diagram:
        \[
          \begin{CD}
             \mathbb  P^n_T @>{h}>> \mathbb  P^n_S \\
          @V{q}VV    @VV{p}V \\
             T   @>{g}>>  S
          \end{CD}
        \]
        \newline
    Suppose that $\mathcal F_T$ is flat over $T$. Then $g^*\mathcal E_m$ is finite locally free on $T$ for every $m \geq m_1$ as we have seen in (*). For an $m \geq m_1$, $g^*\mathcal E_m$ is finite locally free if and only if $g$ factors through $\coprod_e Z^m_{e}$. As we have seen, there is a canonical isomorphism $g^*\mathcal E_m \rightarrow q_*\mathcal F_T(m)$. Let us show that the dimension of $q_*\mathcal F_T(m) \otimes k(t)$ over $k(t)$ is equal to $P_i(m)$ where $s = g(t) \in S_i.$
    
    First, we have a commutative diagram:
        \[
          \begin{CD}
             \mathbb  P^n_{k(t)} @>>> \mathbb  P^n_{k(s)} \\
          @VVV    @VVV \\
             \operatorname{Spec} k(t)   @>>>  \operatorname{Spec} k(s)
          \end{CD}
        \]
    The bottom horizontal morphism is flat, so we can use the flat base change theorem {[Stacks 02KH]} and $H^1(\mathbb P^n_{k(t)}, \mathcal F_T(m)_t) = H^1(\mathbb P^n_{k(s)}, \mathcal F(m)_s) \otimes k(t) = 0$. By (\ref{cohomology_base_change}) and the flat base change theorem we get $q_*\mathcal F_T(m) \otimes k(t) \simeq H^0(\mathbb P^n_{k(t)}, \mathcal F(m)_t) = H^0(\mathbb P^n_{k(s)}, \mathcal F(m)_s) \otimes k(t)$, which has the dimension $P_i(m)$.
    
    Hence $g$ factors through $\coprod_i S_i$. The converse is trivial. $\Box$
    
    \section{Grassmannian}
    Let $0 \leq k \leq n.$ We define the Grassmannian functor
    $$\operatorname{Gr}(n,k)(S) = \{\,  {\rm all \ equivalence \ classes} \ q : \mathcal O_S^{\oplus n} \rightarrow \mathcal F \, {\rm with \ } \mathcal F {\rm \ locally \ free \ of \ rank \ } n - k \,\},$$
    $$\operatorname{Gr}(n,k)(f : S' \rightarrow S) = f^*$$
    
    Here $q : \mathcal O_S^{\oplus n} \rightarrow \mathcal F$ and $q' : \mathcal O_S^{\oplus n} \rightarrow \mathcal F'$ is equivalent if there is an isomorphism $f : \mathcal F \rightarrow \mathcal F'$ such that $f \circ q = q'.$
    \newline
    
    For $I = {i_1, \cdots, i_{n-k}}$ with $1 \leq i_1 < \cdots < i_{n-k} \leq n$, we define $s_I : \mathcal O_S^{\oplus n} \rightarrow \mathcal O_S^{\oplus n-k}$ by $s_I(e_k) = e_{i_k}$. Let $F_I(S) = \{ q \in \operatorname{Gr}(n,k)(S) : q \circ s_I {\rm \ is \ surjective \ (hence \ isomorphic) } \}$. This is a sub-functor of $F = \operatorname{Gr}(n,k)$ and $F_I \subseteq F$ is represented by an open immersion.
    
    According to {[Stacks 089T]}, $F$ is a representable functor. Let $G(n,k)$ be the scheme representing $F$. $G(n,k)$ is called the Grassmannian scheme. Each $F_I$ is represented by an open subscheme $U_I$ of $G(n,k)$ and $U_I \simeq \mathbb A^{n(n-k)}_{\mathbb Z}$'s cover $G(n,k)$. From this fact we know that $G(n,k)$ is smooth over $\mathbb Z$ and the dimension of the fiber over $\mathbb Z$ is $n(n-k).$
    
    It is also known that $G(n,k)$ is projective. Let $\displaystyle A = \bigwedge^{n-k} \mathbb Z^{\oplus n}$ be an Ableian group, and let $P = \operatorname{Proj}_{\mathbb Z} \operatorname{Sym}(A).$
    
    $P$ represents the functor
    
    $S \mapsto $ equivalence classes of surjective morphism $\displaystyle \bigwedge^{n-k} \mathcal O_S^{\oplus n} \rightarrow \mathcal L$ with $\mathcal L$ invertible.
    
    Therefore we can define the morphism of functors $\Phi : F \rightarrow h_P = \operatorname{Hom}(\cdot, P)$ by
    
    $$\displaystyle (q : \mathcal O_S^{\oplus n} \rightarrow \mathcal F) \mapsto (q : \bigwedge^{n-k} \mathcal O_S^{\oplus n} \rightarrow \bigwedge^{n-k} \mathcal F).$$
    
    When we restrict $\Phi$ to the subfunctor $F_I$, this is injective by an easy argument. This means that if we let $f : G(n,k) \rightarrow P$ be the morphism corresponding to $\Phi$, $f|U_I$ is a monomorphism. Since $U_I$'s cover $G(n,k)$, this means that $f$ is a monomorphism. Since $P \simeq \mathbb P^{N-1}_{\mathcal Z}$ where $N = \binom{n}{n-k}$ and by the fact that $G(n,k)$ is proper over $\mathbb Z$ (whose proof is postponed), by the separated cancelation we know that $f$ is proper. A proper monomorphism is a closed immersion; hence $f$ is a closed immersion and $G(n, k)$ is projective.
    
    \section{Hilbert scheme and Quot scheme}
    The Grassmannian $G(k,n)$ is, unformally speaking, a scheme that parametrizes all the free subgroups of $\mathbb Z^n$ of rank $k$. If we base change to some field $K$, then it is just the Grassmannian in the context of geometry.
    \newline
    
    The Hilbert scheme is a scheme that parametrizes all the closed subschemes $Z \subseteq \mathbb P^n_S$ that is flat over a fixed base scheme $S$. Use the functor, we can say that the Hilbert scheme is a scheme that represents the functor $T \mapsto $ closed subschemes $Z \subseteq \mathbb P^n_T$ which is flat over $T$.
    \newline
    
    Combining these two functors, we can define the following functor. Let $X \rightarrow S$ be a projective morphism of Noetherian schemes.
    
    $$\operatorname{Quot}_{X/S, \mathcal E}(T) = \{\,{\rm isomorphism \ classes \ of \ surjections \ } q : \mathcal E_T \rightarrow \mathcal F {\rm \ with} \ \mathcal F {\rm \ flat \ over \ } T \, \},$$
    $$\operatorname{Quot}_{X/S, \mathcal E}(f : T' \rightarrow T) = f_X^*.$$
    
    Here $\mathcal E_T$ is the pull-back of $\mathcal E$ by the fiber change $X\times_ST \rightarrow T$.
    \newline
    
    The Hilbert functor is:
    
    $$\operatorname{Hilb}_{X/S}(T) = \{\,{\rm closed\ subschemes\ of\ } X \times_S T {\rm \ flat \ over \ } T \, \},$$
    $$\operatorname{Hilb}_{X/S, \mathcal E}(f : T' \rightarrow T) = f_X^*.$$
        
    It is obvious that $\operatorname{Hilb}_{X/S} = \operatorname{Quot}_{X/S, \mathcal O_X}$ and that $\operatorname{Gr}(n, k)$ is a sub-functor of $\operatorname{Quot}_{\operatorname{Spec} \mathbb Z/\operatorname{Spec} \mathbb Z, \mathcal O_{\operatorname{Spec} \mathbb Z}^{\oplus n}}$.
    
    Our main topic is the representability of the $\operatorname{Quot}$ functor.
    \newline
    
    {[Insert the division of functors by the Hilbert polynomials]}
    
    \section{Castelnuovo-Mumford Regularity}
    The regularity for a coherent sheaf on a projective space is introduced in the book of Mumford {[Mum]}. We fix a Noetherian ring $R$.
    
    \begin{theorem}
        Let $\mathcal F$ be a coherent sheaf on the projective space $\mathbb P^n_R$. $\mathcal F$ is $m$-regular if $H^i(\mathbb P^n_R, \mathcal F(m-i)) = 0$ for all $i > 0.$
    \end{theorem}
    
    As a simple remark, by the Serre's theorem every coherent sheaf is $m$-regular for some sufficiently large $m$.
    
    \begin{theorem}
        $m$-regularity impiles $m+1$-regularity.
    \end{theorem}
    \begin{proof}
        Induction on $n$. $n = 0$ is obvious since every coherent sheaf is $m$-regular for all $m$. Before we go on inductive step, note that if $R \rightarrow R'$ is flat, then $H^i(\mathbb P^n_R', \mathcal F\otimes_R R') = H^i(\mathbb P^n_R, \mathcal F) \otimes_R R'$ by the flat base change theorem. We can change $R$ under this condition in order to check the regularity. If $R$ is finite, we base change to $R[x]$ to assume that $R$ is infinite.
        
        Let $n > 0$. Since $\mathcal F$ is a coherent sheaf on a Noetherian scheme $P^n_R$, there are only finitely many associated primes of $\mathcal F.$ Let $\mathfrak p_1, \cdots, \mathfrak p_r$ be the corresponding prime ideals in the graded $R$-algebra $S = R[x_0, \cdots, x_n]$. If all the linear form is contained in some $\mathfrak p_i$, i.e., $S_1 \subseteq \mathfrak p_1 \cup \cdots \cup \mathfrak p_r$, then letting $M_i = \mathfrak p_i \cap S_1$, $M_1, \cdots, M_r$ are finite $R$-modules and $S_1 = M_1 \cup \cdots \cup M_r$
    \end{proof}
    
    \begin{theorem}
        If $\mathcal F$ is $m$-regular, then the canonical map
        $$H^0(\mathbb P^n_R, \mathcal F(m)) \otimes H^0(\mathbb P^n_R, \mathcal O(1)) \rightarrow H^0(\mathbb P^n_R, \mathcal F(m+1))$$
    \end{theorem}
    \begin{proof}
    \end{proof}

    \begin{theorem}
        For all $n \geq 0$, there is a polynomial $F(x_0, \cdots, x_n) \in \mathbb Q[x]$ with the following property:
        
        for a coherent ideal sheaf $\mathcal I \subseteq \mathcal O_{\mathbb P^n_R}$ and $a_0, \cdots, a_n$ defined by $$\displaystyle \chi(\mathcal I(m)) = \sum_{i = 0}^{n} a_i\binom{m}{i},$$
        $\mathcal I$ is $F(a_0, \cdots, a_n)$-regular.
    \end{theorem}
    \begin{proof}
        
    \end{proof}
    
    \section{References}
    \begin{enumerate}[]
        \item {[Har]} Hartshorne, R. {\it Algebraic Geometry} (Springer)
        \item {[Mat]} Matsumura, H. {\it Commutative Ring Theory} (Cambridge University Press, Cambridge)
        \item {[Mum]} Mumford, D. {\it Lectures on Curves on an Algebraic Surface} (Princeton University Press, United States of America)
        \item {[Stacks]} The Stacks Project (2021, November 09) https://stacks.math.columbia.edu/
    \end{enumerate}

\end{document}